\appendix

\chapter{Supporting code}
\label{chp:app:supporting-code}

\begin{lstlisting}[language=bash,float=ht,caption={Disassembly excerpt of the 64 bit binary compiled from the code in \cref{lst:local-global-buffer} with \texttt{gcc -o local-global-buffer -O1 local-global-\linebreak[0]buffer.c}, retrieved with \texttt{objdump -D -{}-no-show-raw-insn local-\linebreak[0]global-buffer}}, label={lst:local-global-buffer-disassembly-non-fortified}]
0000000000001169 <main>:
1169:       push   %rbx
116a:       sub    $0x30,%rsp
116e:       mov    %fs:0x28,%rax
1177:       mov    %rax,0x28(%rsp)
117c:       xor    %eax,%eax
117e:       lea    0xe83(%rip),%rdi
1185:       callq  1030 <puts@plt>
118a:       mov    $0x100,%edx
118f:       lea    0x2eea(%rip),%rsi
1196:       mov    $0x0,%edi
119b:       callq  1060 <read@plt>
11a0:       lea    0xe89(%rip),%rdi
11a7:       callq  1030 <puts@plt>
11ac:       mov    %rsp,%rbx
11af:       mov    $0x100,%edx
11b4:       mov    %rbx,%rsi
11b7:       mov    $0x0,%edi
11bc:       callq  1060 <read@plt>
11c1:       mov    %rbx,%rdx
11c4:       lea    0x2eb5(%rip),%rsi
11cb:       lea    0xe86(%rip),%rdi
11d2:       mov    $0x0,%eax
11d7:       callq  1050 <printf@plt>
11dc:       mov    0x28(%rsp),%rax
11e1:       sub    %fs:0x28,%rax
11ea:       jne    11f7 <main+0x8e>
11ec:       mov    $0x0,%eax
11f1:       add    $0x30,%rsp
11f5:       pop    %rbx
11f6:       retq
11f7:       callq  1040 <__stack_chk_fail@plt>
\end{lstlisting}

\begin{lstlisting}[language=bash,float=ht,caption={Disassembly excerpt of the 64 bit binary compiled from the code in \cref{lst:local-global-buffer} with \texttt{gcc -o local-global-buffer -D\_FORTIFY\_SOURCE=2 -O1 local-global-buffer.c}, retrieved with \texttt{objdump -D -{}-no-show-\linebreak[0]raw-insn local-global-buffer} (note the call to \texttt{\_\_read\_chk} instead of \texttt{read} in line 20)}, label={lst:local-global-buffer-disassembly-fortified}]
0000000000001179 <main>:
1179:       push   %rbx
117a:       sub    $0x30,%rsp
117e:       mov    %fs:0x28,%rax
1187:       mov    %rax,0x28(%rsp)
118c:       xor    %eax,%eax
118e:       lea    0xe73(%rip),%rdi
1195:       callq  1040 <puts@plt>
119a:       mov    $0x100,%edx
119f:       lea    0x2eda(%rip),%rsi
11a6:       mov    $0x0,%edi
11ab:       callq  1060 <read@plt>
11b0:       lea    0xe79(%rip),%rdi
11b7:       callq  1040 <puts@plt>
11bc:       mov    %rsp,%rbx
11bf:       mov    $0x20,%ecx
11c4:       mov    $0x100,%edx
11c9:       mov    %rbx,%rsi
11cc:       mov    $0x0,%edi
11d1:       callq  1030 <__read_chk@plt>
11d6:       mov    %rbx,%rcx
11d9:       lea    0x2ea0(%rip),%rdx
11e0:       lea    0xe71(%rip),%rsi
11e7:       mov    $0x1,%edi
11ec:       mov    $0x0,%eax
11f1:       callq  1070 <__printf_chk@plt>
11f6:       mov    0x28(%rsp),%rax
11fb:       sub    %fs:0x28,%rax
1204:       jne    1211 <main+0x98>
1206:       mov    $0x0,%eax
120b:       add    $0x30,%rsp
120f:       pop    %rbx
1210:       retq
1211:       callq  1050 <__stack_chk_fail@plt>
\end{lstlisting}

\lstinputlisting[language=C,float=ht,caption={C program outputting an environment variable's memory location in a target program {\cite[147\psq]{Erickson2008}}},label={lst:getenvaddress}]{code/getenvaddress.c}

%%% Local Variables:
%%% mode: latex
%%% TeX-master: "Report"
%%% End:
