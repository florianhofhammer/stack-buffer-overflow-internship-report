\newenvironment{abstractpage}
  {\cleardoublepage\thispagestyle{empty}\chapter*{\abstractname}}
  {\cleardoublepage}
\renewenvironment{abstract}[1]
  {\bigskip\begin{otherlanguage}{#1}%
   \begin{center}\bfseries\abstractname\end{center}\noindent}
  {\end{otherlanguage}\par\bigskip}

\begin{abstractpage}
  \begin{abstract}{american}
    The goal of the presented paper is to summarize different stack buffer overflow attack and defense methods in a single paper.
    In this context, the question arises how some defense methods can be bypassed by an attacker and how a defender can improve on the defense mechanisms to thwart stack buffer overflow based attacks.

    To achieve these goals and answer these questions, several attack methods and their implications on the security of currently deployed defense mechanisms are presented.
    They show that even on modern systems, stack buffer overflow vulnerabilities can be exploited to execute arbitrary code controlled by an attacker under specific circumstances.
    Although stack buffer overflows are hard or impossible to exploit depending on the circumstances of the vulnerability and the system configuration, the need for security measure enhancements arises.
    Such improvements and expansions on current defense mechanisms against stack buffer overflows are also presented and compared with regard to their performance implications on a system.
  \end{abstract}
  \vspace{2em}
  \begin{abstract}{french}
    L'objectif de l'article présenté est de résumer en un seul document les différentes méthodes d'attaque et de défense contre les débordements de pile.
    Dans ce contexte, la question se pose de savoir comment certaines méthodes de défense peuvent être contournées par un attaquant et comment un défenseur peut améliorer les mécanismes de défense pour contrecarrer les attaques basées sur le débordement de pile.

    Pour atteindre ces objectifs et répondre à ces questions, plusieurs méthodes d'attaque et leurs implications sur la sécurité des mécanismes de défense actuellement déployés sont présentées.
    Elles montrent que même sur les systèmes modernes, les vulnérabilités liées au débordement de pile peuvent être exploitées pour exécuter du code arbitraire contrôlé par un attaquant dans des circonstances spécifiques.
    Bien que les débordements de pile soient difficiles ou impossibles à exploiter selon les circonstances de la vulnérabilité et la configuration du système, il est nécessaire d'améliorer les mesures de sécurité.
    Ces améliorations et extensions des mécanismes de défense actuels contre les débordements de pile sont également présentées et comparées en ce qui concerne leurs implications sur les performances d'un système.
  \end{abstract}
\end{abstractpage}

%%% Local Variables:
%%% mode: latex
%%% TeX-master: "Report"
%%% End:
