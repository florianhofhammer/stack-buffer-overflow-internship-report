\chapter{Background}
\label{chp:background}

In this chapter, some of the basic technical background to buffer overflows is explained.
\Cref{sec:basic-vocabulary,sec:system-model} provide basic definitions concerning wording and system model used throughout the rest of the paper.
\Cref{sec:process-memory} explains the basic memory layout of a process running on the Linux kernel.
\Cref{sec:stack-setup-and-usage} focuses on how the stack is used during program execution.
\Cref{sec:stack-buffer-overflows} finally presents what a stack buffer overflow is, how it can be exploited and why it is so dangerous.

\section{Basic vocabulary}
\label{sec:basic-vocabulary}
As this report is about stack buffer overflows, it is important to know what this wording means.
The \emph{stack} is a data structure used in modern \acs{os}'s memory management and is explained in \cref{sec:process-memory,sec:stack-setup-and-usage}.

A \emph{buffer} generally is a ``limited, contiguously allocated set of memory'' \cite[12]{Anley2007}.
In this report, we will often refer to C programming language's \texttt{char} arrays as buffers.
Such arrays are fulfilling the above definition and are generally used to hold \gls{ascii} encoded text or arbitrary binary data, as a single \texttt{char} in C refers to exactly one byte of data.

\section{System model}
\label{sec:system-model}

In the cases where references to the Linux kernel, the \gls{glibc} or the \gls{gcc} are made, they refer to specific versions of those software products.
In later versions, the behavior might change in order to mitigate some of the weaknesses described in this report.

Those specific software versions comprise version 5.4.50 of the Linux kernel%
\footnote{Source code can be viewed online at \href{https://git.kernel.org/stable/h/v5.4.50}{git.kernel.org} \cite{LKD2020}}%
.
This release is a \gls{lts} release which means that it will be maintained until the end of 2025 \cite{LKO2020}.
The version of \gls{glibc} is 2.31%
\footnote{Source code can be viewed online at \href{https://sourceware.org/git/?p=glibc.git;a=tree;h=6ee690ef6fa36bf79d2e05b5a30a4f7e10ba3937;hb=9ea3686266dca3f004ba874745a4087a89682617}{sourceware.org}}%
.
\gls{gcc} is used in in the latest 9.x line version 9.3.0.

At the time of writing this report, newer stable releases of the Linux kernel and \gls{gcc} are already available (5.7.x line for the Linux kernel \cite{LKO2020a}, 10.x line for \gls{gcc} \cite{FSF2020}).
However, this does not impose any problems on the real-world consequences of this report's contents.
The newest editions of popular Linux distributions for \texttt{x86\_64/amd64} processor architectures like Red Hat Enterprise Linux (version 8.2), Ubuntu (version 20.04 \gls{lts}), Debian (version 10.4) are based on the Linux 5.4 \gls{lts} kernel line or even older versions%
\footnote{See also discussion on \href{https://www.reddit.com/r/webhosting/comments/beg0z0/should_i_use_an_lts_version_of_ubuntu_for_my_web/}{reddit.com} on whether to use \gls{lts} versions for production server deployment}
\cite{RedHat2020,Canonical2020,SPI2020,SPI2020a}.
Also, the highest kernel version used for development of the Android mobile \gls{os}, which has a market share of about 70\% \cite{Statcounter2020}, is Linux 5.4 \gls{lts} \cite{GoogleLLC2020}.
Thus, it is safe to assume that the majority of devices running Linux kernels or kernels closely derived from the Linux kernel run version 5.4.50 or lower.

A similar observation can be made for \gls{gcc} version 9.3.0.
The aforementioned Linux distributions all by default ship the \gls{gcc} package in version 9.3.0 or below \cite{RedHat2020a,Canonical2020a,SPI2020b}.
It can therefore be safely assumed that \gls{gcc} version 9.3.0 is still very widespread and in frequent use.

If not otherwise stated, the following chapters and sections are referring to a 64 bit (\texttt{x86\_64} or \texttt{amd64} architecture) \acs{gnu}/Linux system with the aforementioned software versions.
Thus, if referring to processor registers, by default the corresponding 64 bit register names (starting with the letter \texttt{r} instead of \texttt{e} on 32 bit \texttt{x86} architecture, for example \texttt{rsp} instead of \texttt{esp}) are used.
The 32 bit register names are only used if explicitly referring to 32 bit code.
In addition, a memory layout equivalent to the memory layout for C programs as described in \cref{sec:process-memory} should be assumed.

The terms \texttt{x86} and \texttt{i386} for 32 bit processor architectures and \texttt{x86\_64} and \texttt{amd64} for 64 bit processor architectures are used synonymously throughout this report.

This report only targets userspace stack buffer overflows.
The kernel in the system model in use is assumed to be safe in a sense that it has no stack buffer overflow vulnerabilities and is thus not targeted in this report.

\section{A process's memory}
\label{sec:process-memory}

When a program is executed, it has to be loaded into the computer's main memory before the kernel can transfer control to the newly-created process.
During that process, several sections of the \gls{elf} binary%
	\footnote{See e.g. the \href{https://en.wikipedia.org/wiki/Executable_and_Linkable_Format}{wikipedia.org} entry for further information on \gls{elf}}
are put straight into memory.
For example, the \texttt{.text} section containing the actual binary instructions for the program or the \texttt{.data} and \texttt{.bss} sections holding initialized and uninitialized global variables, respectively, are allocated at the lower end of the process's virtual memory space.
The size of those sections is already known on program initialization which is why they can be allocated without the need of spare memory to grow those memory regions.

\begin{figure}[htb]
	\centering
	\includegraphics[width=0.5\textwidth]{figures/memory-layout}
	\caption{The memory layout of a Linux process \cite[804]{Bryant2011}}
	\label{fig:memory-layout}
\end{figure}

As \cref{fig:memory-layout} shows, above those memory regions the dynamically growing and shrinking memory regions like the heap, the mapped memory and the stack are located.
Those regions have in common that depending on the program's execution flow, their size changes.
Thus, they need to have free memory around them in order to be able to dynamically grow (marked in blue in \cref{fig:memory-layout}).

The heap contains data which has to be allocated dynamically because the necessary amount of memory is not fixed and has to be determined during runtime.
This includes for example arrays or buffers like the \texttt{input} buffer in \cref{lst:memory-layout}.
As the user input cannot be determined before running the program, the buffer has to be allocated dynamically which happens with a call to \texttt{malloc} in line 11 of the listing.
The contents of the buffer are then stored on the heap.

The stack on the other hand contains data with fixed size.
In the example of \cref{lst:memory-layout}, sizes of variables like \texttt{int max\_len} or \texttt{char *input} are known at compile time.
Thus, they can be stored on the stack by increasing the stack size by a fixed size when program flow enters the corresponding function.
It has to be noted here that even though the contents of the \texttt{input} buffer are stored in the heap, the pointer to those contents is stored on the stack.

The mapped memory is a region where files from the computers file system are mapped to directly address those files' contents just like any other memory contents like variables or program instructions.
Such files can be any files accessible by the running program.
The main usage for this feature is to map libraries into memory, for example \gls{glibc} or custom created libraries.

\lstinputlisting[language=C,float=ht,caption={C program containing stack variables as well as heap variables},label={lst:memory-layout}]{code/memory-layout.c}

\section{Stack setup and usage}
\label{sec:stack-setup-and-usage}

As shown in \cref{sec:process-memory}, the stack can contain program data with a size already known at compile time.
It is a data structure behaving according to the \gls{lifo} principle meaning that the data on top of the stack is the data that was put on the stack last and will be removed from the stack first with regard to data further down on the stack.
When a function is entered, the stack size is increased by the accumulative size of the function's variables by decrementing the stack pointer by this size.
This is also known as ``pushing a stack frame onto the stack'' where \emph{stack frame} refers to the totality of a function's local data.
The stack pointer is stored in the processor register \texttt{esp} (\texttt{x86}/\texttt{i386} architecture) or \texttt{rsp} (\texttt{x86\_64}/\texttt{amd64} architecture) and points to the top of the stack.
On leaving the function, the stack size is decreased by the same amount of memory by incrementing the stack pointer, also known as ``popping a stack frame from the stack''.
This implies that variables declared in a function can only be safely accessed as long as the function is active.
\emph{Active} here means that either control flow is currently inside this function or inside another function called by this function, maybe even recursively.
Otherwise, the memory where those variables reside was already freed by decreasing the stack size and any access to such variables yields nondeterministic behavior.

When a function is called, not only its variables are stored on the stack but also some control flow information like the \gls{rip}.
The \gls{rip} points to the next instruction in the calling function which should be executed after the called function returns.
The \texttt{call} and \texttt{ret} assembly instructions manage this pointer.
\texttt{call} pushes the address of the next instruction onto the stack and then enters the called function.
In the called function, \texttt{ret} at the end of the function pops the saved address from the stack and continues execution at this address.

Depending on the compiler configuration, there is also often a \gls{sfp} put on the stack which usually lies between the \gls{rip} and the called function's local data on the stack.
Apart from the \texttt{esp/rsp} register, the \texttt{ebp/rbp} register can be used to address stack contents.
At the start of a function, the current \texttt{ebp/rbp} register content is pushed onto the stack as the \gls{sfp} and the \texttt{esp/rsp} content is loaded into \texttt{ebp/rbp}.
Throughout the function, this register's value does not change whereas the \texttt{esp/rsp} value can change through \texttt{push} or \texttt{pop} instructions as the stack pointer should always point to the top of the stack.
The static value of the \texttt{ebp/rbp} register thus allows to address a function's stack content by fixed offsets that don't change throughout the function's code and is often referred to as \emph{frame pointer}.
On returning from a function, the \gls{sfp} is loaded into \texttt{ebp/rbp} to allow the caller function to use its static offsets into its own stack frame again when continuing its execution.

Thus, the stack not only contains data but also important control flow information.

\section{Stack Buffer Overflows}
\label{sec:stack-buffer-overflows}

This mixture of data and control flow information is exactly where problems arise.

Under the assumption that a function exists somewhere in a program as shown in \cref{lst:vulnerable-function}, the stack layout of such a function may look as shown in \cref{fig:stack-layout-without-data}.
After the \gls{rip}, the compiler by default usually positions the \gls{sfp} and the local variables in the order they were declared%
	\footnote{
		The layout as shown in \cref{fig:stack-layout} should just be seen as an example, as it can differ depending on compiler behavior or processor architecture (here: 32 bit architecture with stack alignment on 4 bytes, function arguments passed on the stack).
	}%
.

\begin{lstlisting}[language=C,float=ht,caption={C function with buffer overflow vulnerability}, label={lst:vulnerable-function}]
void copy(char *bar) {
    char c[12];
    strcpy(c, bar);
}
\end{lstlisting}

A stack buffer overflow vulnerability here lies in the call to \texttt{strcpy} in line 6.
This function copies a string from the memory location where \texttt{bar} points to to the memory location where \texttt{c} points to.
However, it copies data until it encounters a \texttt{0x00} byte which marks the end of the string, regardless of whether the destination can take all the copied data.

If the string to copy is strictly less than 12 bytes, no problem occurs as the buffer can hold the string (11 bytes at maximum) and the string terminating \texttt{0x00} byte as shown in \cref{fig:stack-layout-no-overflow}.

If the string to copy is 12 bytes or more, a buffer overflow occurs as shown in \cref{fig:stack-layout-overflow}.
This means that the \texttt{strcpy} function copies 12 bytes or more of data to the buffer \texttt{c} even though it cannot hold more than 12 bytes.
Including the terminating \texttt{0x00} byte, this string copy operation thus overwrites data residing on the stack after the destination buffer.

\begin{figure}[htb]
	\centering
	\begin{subfigure}[t]{0.28\textwidth}
		\centering
		\includegraphics[height=0.25\textheight]{figures/Stack_Overflow_2}
		\caption{Stack layout with control flow information and local data \cite{Lynn2007}}
		\label{fig:stack-layout-without-data}
	\end{subfigure}
	\hfill
	\begin{subfigure}[t]{0.28\textwidth}
		\centering
		\includegraphics[height=0.25\textheight]{figures/Stack_Overflow_3}
		\caption{Stack contents after copying data to the \texttt{char} array without overflowing it \cite{Lynn2007a}}
		\label{fig:stack-layout-no-overflow}
	\end{subfigure}
	\hfill
	\begin{subfigure}[t]{0.28\textwidth}
		\centering
		\includegraphics[height=0.25\textheight]{figures/Stack_Overflow_4}
		\caption{Stack contents after copying data to the \texttt{char} array and overflowing it \cite{Lynn2007b}}
		\label{fig:stack-layout-overflow}
	\end{subfigure}
	\caption{Stack layout of a function with two local variables: a 12 byte buffer (\texttt{char c[12]}) and a pointer (\texttt{char *bar})}
	\label{fig:stack-layout}
\end{figure}
\todo{If enough time, edit all graphics to represent 64 bit stacks}

If an attacker can control the data which is copied onto the stack, they%
	\footnote{In this paper, the gender-neutral singular pronoun ``they'' and pronouns derived from it are used instead of singular ``he'', ``she'' or pronouns derived from those if no specific gender is provided for a person.}
can carefully craft an input string so that they can overwrite the \gls{sfp} or \gls{rip} with attacker-controlled data and thus control where the program tries to look for function-local data (overwritten \gls{sfp}) or to continue execution (overwritten \gls{rip}) after it returns from the current function.
A stack buffer overflow always occurs if a buffer just like \texttt{c} in \cref{lst:vulnerable-function} is filled with more data than it can hold.
This does not necessarily mean that the \gls{sfp} or \gls{rip} is overwritten by a stack buffer overflow, but in most cases this is the goal of an attacker in order to divert control flow in the program's execution or manipulate a function's data.

How an attacker can divert control flow or manipulate data and what implications come with this possibility is explained in \cref{chp:attack-vectors}.

