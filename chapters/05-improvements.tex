\chapter{Improvements for defense mechanisms}
\label{chp:defense-mechanism-improvements}

In this chapter, we refer to weaknesses in current security mechanisms that were described in \cref{chp:current-defense-mechanisms,chp:attack-vectors} and present several suggestions on how to improve security in regards to stack buffer overflow exploit prevention.

In \cref{sec:aslr-improvements,sec:stack-canary-improvements} we refer to mechanisms mainly based on randomization and the problems coming with such randomization.
In \cref{sec:function-pointer-protection-improvements,sec:control-flow-integrity} we then present more deterministic approaches to protect processes against stack buffer overflow exploits.
Finally, \cref{sec:performance-comparison} concludes with comparing the performance impact of the previously presented security improvements.

\section{\glsentrylong{aslr} (\glsentryshort{aslr})}
\label{sec:aslr-improvements}

\begin{itemize}
	\item{Randomized on program startup}
	\item{Clone-probing attacks: no new randomization on \texttt{fork}}
\end{itemize}

\section{Stack canaries}
\label{sec:stack-canary-improvements}

\begin{itemize}
	\item{Random canary created on program startup}
	\item{Same canary for all functions $\Rightarrow$ canary disclosure in one function makes all other functions vulnerable}
	\item{Clone-probing attacks: no new randomization on \texttt{fork}}
\end{itemize}

\section{Function pointer protection}
\label{sec:function-pointer-protection-improvements}

\begin{itemize}
	\item{Encryption}
	\item{Mangling}
	\item{Stored in protected memory regions}
\end{itemize}

\section{\glsentrylong{cfi} (\glsentryshort{cfi})}
\label{sec:control-flow-integrity}

\begin{itemize}
	\item{\gls{cfi} => overhead?}
	\item{Intel CET $\Rightarrow$ shadow stack, branch validation}
\end{itemize}

\section{Performance comparison of presented approaches}
\label{sec:performance-comparison}

\begin{itemize}
	\item{Table with performance overheads}
	\item{Gather information from different papers}
\end{itemize}