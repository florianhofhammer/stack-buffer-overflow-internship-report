\chapter{Improvements for defense mechanisms}
\label{chp:defense-mechanism-improvements}

\section{\glsentrylong{aslr} (\glsentryshort{aslr})}
\label{sec:aslr-improvements}

\begin{itemize}
	\item{Randomized on program startup}
	\item{Clone-probing attacks: no new randomization on \texttt{fork}}
\end{itemize}

\section{Stack canaries}
\label{sec:stack-canary-improvements}

\begin{itemize}
	\item{Random canary created on program startup}
	\item{Same canary for all functions $\Rightarrow$ canary disclosure in one function makes all other functions vulnerable}
	\item{Clone-probing attacks: no new randomization on \texttt{fork}}
\end{itemize}

\section{\glsentrylong{cfi} (\glsentryshort{cfi})}
\label{sec:control-flow-integrity}

\begin{itemize}
	\item{\gls{cfi} => overhead?}
	\item{Intel CET $\Rightarrow$ shadow stack, branch validation}
\end{itemize}

\section{Performance comparison of mentioned approaches}
\label{sec:performance-comparison}

\begin{itemize}
	\item{Table with performance overheads}
	\item{Gather information from different papers}
\end{itemize}

% Depending on time left!
\chapter{OS assessment}
\label{chp:os-assessment}

\begin{itemize}
	\item{Scan different Linux distributions' /bin directory for insecure applications}
	\item{Example: \texttt{snapctl} on Ubuntu 20.04: no stack canaries, no position independent executable, no \gls{relro}}
\end{itemize}
%

\chapter{Related work and further reading}
\label{chp:related-work}

\begin{itemize}
	\item{Compiler (optimizations)}
	\item{Heap overflows}
	\item{String format vulnerability}
\end{itemize}

\chapter{Conclusion}
\label{chp:conclusion}