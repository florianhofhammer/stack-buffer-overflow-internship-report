\chapter{Introduction}
\label{chp:introduction}

\todo[inline]{Basic introduction here}

\section{Basic vocabulary}
As this report is about stack buffer overflows, it is important to know what this wording means.
The \emph{stack} is a data structure used in modern \gls{os} memory management and is explained in \cref{sec:process-memory,sec:stack-setup-and-usage}.

A \emph{buffer} generally is a ``limited, contiguously allocated set of memory'' \cite[12]{Anley2007}.
In this report, we will often refer to C programming language's \texttt{char} arrays as buffers.
Such arrays are fulfilling the above definition and are generally used to hold \gls{ascii} encoded text or arbitrary binary data, as a \texttt{char} in C refers to exactly one byte of data.

\section{System model}
\label{sec:system-model}

In the cases where references to the Linux kernel, the \gls{glibc} or the \gls{gcc} are made, they refer to specific versions of those software products.
In later versions, the behavior might change in order to mitigate some of the weaknesses described in this report.

Those specific software versions comprise version 5.4.44 of the Linux kernel%
	\footnote{Source code can be viewed online at \href{https://git.kernel.org/stable/h/v5.4.44}{git.kernel.org} \cite{LKD2020}}%
.
This release is a \gls{lts} release which means that it will be maintained until the end of 2025 \cite{LKO2020}.
The version of \gls{glibc} is 2.31%
	\footnote{Source code can be viewed online at \href{https://sourceware.org/git/?p=glibc.git;a=tree;h=6ee690ef6fa36bf79d2e05b5a30a4f7e10ba3937;hb=9ea3686266dca3f004ba874745a4087a89682617}{sourceware.org}}%
.
\gls{gcc} is used in in the latest 9.x line version 9.3.0.

At the time of writing this report, newer stable releases of the Linux kernel and \gls{gcc} are already available (5.7 for the Linux kernel \cite{LKO2020a}, 10.1 for \gls{gcc} \cite{FSF2020}).
\todo{Look up whether the referenced parts changed at all or if everything's still valid for the new versions}
However, this does not impose any problems on the real-world consequences of this report's contents.
The newest editions of popular Linux distributions for \texttt{x86\_64/amd64} processor architectures like Red Hat Enterprise Linux (version 8.2), Ubuntu (version 20.04 \gls{lts}), Debian (version 10.4) are based on the Linux 5.4 \gls{lts} kernel line or even older versions%
	\footnote{See also discussion on \href{https://www.reddit.com/r/webhosting/comments/beg0z0/should_i_use_an_lts_version_of_ubuntu_for_my_web/}{reddit.com} on whether to use \gls{lts} versions for production server deployment}
\cite{RedHat2020,Canonical2020,SPI2020,SPI2020a}.
Also, the highest kernel version used for development of the Android mobile \gls{os}, which has a market share of about 70\% \cite{Statcounter2020}, is Linux 5.4 \gls{lts} \cite{GoogleLLC2020}.
Thus, it is safe to assume that the majority of devices running Linux kernels or kernels closely derived from the Linux kernel run version 5.4.44 or lower.

A similar observation can be made for \gls{gcc} version 9.3.0.
The aforementioned Linux distributions all by default ship the \gls{gcc} package in version 9.3.0 or below \cite{RedHat2020a,Canonical2020a,SPI2020b}.
It can therefore be safely assumed that \gls{gcc} version 9.3.0 is still very widespread and in frequent use.

If not otherwise stated, the following chapters and sections are referring to a 64 bit (\texttt{x86\_64} or \texttt{amd64} architecture) \acs{gnu}/Linux system with the aforementioned software versions.
Thus, if referring to processor registers, by default the corresponding 64 bit register names (starting with the letter \texttt{r} instead of \texttt{e} on 32 bit \texttt{x86} architecture, for example \texttt{rsp} instead of \texttt{esp}) are used.
The 32 bit register names are only used if explicitly referring to 32 bit code.
In addition, a memory layout equivalent to the memory layout for C programs as described in \cref{sec:process-memory} should be assumed.

The terms \texttt{x86} and \texttt{i386} for 32 bit processor architectures and \texttt{x86\_64} and \texttt{amd64} for 64 bit processor architectures are used synonymously throughout this report.

This report only targets userspace stack buffer overflows.
The kernel is assumed to be safe in a sense that it has no stack buffer overflow vulnerabilities and is thus not targeted in this report.
\todo{Maybe move to other section? Not really defining the system...}

\section{Report structure}
\label{sec:report-structure}

\todo[inline]{Describe structure of the report and what to find in which chapter}
