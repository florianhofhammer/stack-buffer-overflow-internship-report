\chapter{Introduction}
\label{chp:introduction}

When searching the \gls{cve} list \cite{MITRECorporation2020} for vulnerabilities, the search term ``stack buffer overflow'' yields 2608 results with 57 of them discovered in 2020.
Entering only the search term ``buffer overflow'' returns a list of 10945 \gls{cve} entries%
	\footnote{Even though the \gls{cve} search advisory states that such keywords may yield incorrect results, checking random samples of the results showed that the results were in fact correctly corresponding to the given search terms.}%
.
Those high numbers show that this kind of memory vulnerability is still very widespread, even though it has already been presented towards the end of the last century, for example in \citeauthor{AlephOne1996}'s famous paper \citetitle{AlephOne1996} \cite{AlephOne1996}.
\todo{Maybe rework last phrase... technically not 100\% correct}

To the author's best knowledge, no scientific papers providing an overview over stack buffer overflow exploit mechanisms and measures to mitigate such exploits exist.
Thus, the goal of this report is to gather information from scientific work and consolidate and compare the findings of different researchers in a single paper to provide an overview and a collection of ideas for exploit development and mitigation.

This paper is partitioned into several chapters.
Firstly, we want to provide a short introduction into the technical background of memory layout, stack usage and stack buffer overflows in \cref{chp:background}.
Secondly, we describe current protection measures against stack buffer overflows implemented by the compiler, the \gls{os} or other parts of the runtime environment in \cref{chp:current-defense-mechanisms}.
In \cref{chp:attack-vectors}, we present a collection of exploit mechanisms and approaches which are based on stack buffer overflow vulnerabilities.
It is important to mention that despite starting with approaches that are easily mitigated by measures from \cref{chp:current-defense-mechanisms}, also more sophisticated exploit mechanisms are presented which aim to bypass the aforementioned protection measures.
\Cref{chp:defense-mechanism-improvements} in consequence then contains several improvement proposals to the defense mechanisms from \cref{chp:current-defense-mechanisms} which could potentially thwart the even more sophisticated approaches from \cref{chp:attack-vectors}.
Concluding with \cref{chp:related-work,chp:conclusion}, we shed light on the limits and restrictions of this paper, present ideas how to generalize the findings and overcome the limits in possible future work and draw a final conclusion.

The choice of exploit and defense mechanisms in \cref{chp:attack-vectors,chp:defense-mechanism-improvements} was made based on the experiences made during the practical part of the internship this report corresponds to.
During the first half of the internship, focus was lying on implementing different exploit mechanisms as described in \cref{chp:attack-vectors}.
All the test exploits were implemented on a \texttt{x86\_64} machine running a current Linux kernel (cf. \cref{sec:system-model}).
The successful implementation of different exploits leads to the conclusion that additional security measures are needed, even in modern systems.
This report aims to show why this is the case and how this could be achieved.